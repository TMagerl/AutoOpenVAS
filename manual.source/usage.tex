\section{Usage}

\breakpoint\subsection{Run \sw}

You can start \sw from terminal after configuration, though the main purpose is to be run by cronjobs.

For a simple cron job see \tocode{cron.txt}. This example results in scanning for (new) machines and testing next machine every hour.

\codein{cron.txt}{python}{simple cron job sample}

For a more advanced cron job see \tocode{cron_advanced.txt}. Results in scanning for (new) machines every hour, testing every two hours during work time, hourly over night and every 30 minutes on weekends.

\codein{cron_advanced.txt}{python}{advanced cron job sample}

\warn Read and write privileges are needed for import and export paths.

\hint All actions will be submitted to your local syslog server. See \tosec{verbose} for more details.
 

\clearpage
\breakpoint\subsection{Parameters}\label{sec:parameters}

\breakpoint\subsubsection*{-v}\label{sec:verbose}
Use \emph{-v} if you want more details in your syslog.

\emph{-vv} for insane debug spam.

\breakpoint\subsubsection*{-print}\label{sec:print_console}
Use \emph{-print} to print a human readable table to console. See\tosec{print_console}.
Can be combined with \emph{-verify}.

\breakpoint\subsubsection*{-run}
Use \emph{-run} to start the next vulnerability check. The candidates will be rotating, so each time you start \sw with this argument, the next machine in the row will be tested.

After a test is started \sw waits 2 minutes and checks again if task is running. If the task was aborted for any reason it will continue with the next machine.

\breakpoint\subsubsection*{-scan [subnet]} scan the given subnet for machines.

Populate OpenVAS automatically with all machines of your network. See\tosec{active}.

Notation is the network address of the subnet you want to scan including CIDR (e.g. \emph{-scan 192.168.1.0/24}).

\warn \emph{nmap} has to be installed for this to work.

\breakpoint\subsubsection*{-verify}\label{sec:verifymac} verify MAC address before scan with \emph{-verify}.
In case a MAC/IP address pair doesn't match the configured (see\tosec{inventory}) one (any more), the job will be skipped as \emph{failed}.

\warn \emph{arping} has to be installed for this to work.
